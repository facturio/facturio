\documentclass{article}
\usepackage[utf8]{inputenc}
\usepackage{enumitem}

\title{\Huge{Exigences Logicielles {\em Facturio}}}
    \author{\small \textsc{BAZAN} Clément, textsc{BENJELLOUN} Youssef, \and \small \textsc{HAFSAOUI} Théo, \textsc{LOMBARDO} Quentin, \textsc{OLIVIER} Tom}
\date{}

\begin{document}

\maketitle
\tableofcontents
\section{Utilisateur}
    \subsection{Saisir les données de l'utilisateur} \label{ss:sdu}
    Le  système  doit  permettre  à  l’utilisateur de saisir ses données:
    \begin{itemize}[leftmargin=0.55in]
        \item Logo
        \item Nom
        \item Prénom
        \item Nom Entreprise
        \item Adresse
        \item Adresse courriel/mail
        \item Code postale
        \item Numéro de téléphone
        \item Numéro SIRET/SIREN
    \end{itemize}
    \subsection{Enregistrer les données de l'utilisateur}
    Le système doit permettre d'enregistrer les données de l'utilisateur (voir section ~\ref{ss:sdu}) pour un usage récurrent.
    \subsection{Mettre à jour les informations utilisateur}
    Le système peut permettre une modification ultérieur des informations de l'utilisateur



\section{Client}
    \subsection{Créer profil client}\label{ss:cpc}
    
    Le système doit permettre d'enregistrer un profil client sous la forme d'un 
    fichier \emph{XML} (ou autre) au travers d'un formulaire.
    
    Les informations d'un client lorsque c'est un particulier :
    \begin{itemize}[leftmargin=0.55in]
        \item Nom/Prénom (uniquement Nom si entreprise)
        \item Adresse
        \item Adresse internet
        \item Code postale
        \item Numéro de téléphone
    \end{itemize}
    
    Les attributs supplémentaires lorsque les clients sont des sociétés:
    \begin{itemize}[leftmargin=0.55in]
        \item Nom Entreprise
        \item Numéro SIRET/SIREN
    \end{itemize}
    
    \subsection{Consulter profil client}
    Le système doit permettre de lire les informations d'un profil client à l'aide de l'interface graphique.
    
    \subsubsection{Recherche profil client}
    Le système doit permettre la recherche de clients à partir de une ou plusieurs donner.
    Interrogation de la Base de données (Recherche par valeurs, par plages)
    
    \subsection{Mettre à jour profil client}
    Le système doit permettre la modification du profil client,
    Modification des colonnes correspondant à la ligne d'un profil client (dans la base de données
    
    \subsection{Supprimer profil client}
    Le système doit permettre la suppression d'un profil client
    Suppression des lignes correspondant au client
    
    \subsection{Importer des clients}
    Le système doit permettre de importer des clients en format texte et \emph{XML}.
    
    \subsection{Exporter des clients}
    Le système doit permettre de importer des clients en format texte et \emph{XML}.
    
    \subsection{Affichage des clients sur une carte}
    Le système doit permettre l'affichage de la localisation des clients sur une carte 
    à partir des adresses enregistrer dans la base de donnes des clients 


\section{Devis}
    \subsection{Établir devis}
    Le système doit permettre la création d'un devis.
    Le devis est constitué de 3 parties :
    \begin{enumerate}[leftmargin=0.55in]
    \item Une partie contextuelle (services/produits achetés, montant, date, heures, etc)
    \item Une partie prestataire avec ses information et son logo (cf.\ref{ss:sdu})
    \item Une partie client avec ses informations (cf.\ref{ss:cpc})
    \end{enumerate}
    \subsubsection{Saisir informations devis}
    Le système doit permettre de saisir les données  de la facture à partir de
    boîtes de dialogue
    \subsubsection{Charger informations devis}
    Le système doit permettre d'entrer les informations liés au prestataire et au client à partir
    de profil utilisateurs et de profil clients
    \subsubsection{Enregistrement dans la base de données}
    Lors de la création d'un devis le système permet la création d'une entrée dans la base de données
    \subsection{Lire devis}
    \subsubsection{Recherche d'un devis}
    Le système doit permettre d'interroger la base de données pour retrouver
    un devis à partir de ses informations (date, nom client ...)
    \subsection{Mettre à jour devis}
    Le système doit permettre la mise a jour du devis.
    \subsection{Exporter devis}
    Le système doit permettre de sortir le devis sous forme de \emph{PDF}
    \subsection{Appliquer style à devis}
    Le système permet d'appliquer différents styles au devis
    
    \subsection{Transformer un devis en facture}
    Le système doit permettre de transformer un devis en facture.
\section{Factures}
    \subsection{Établir facture}
    Le système doit permettre la création d'une facture.
    La facture est constituée de 3 parties :
    \begin{enumerate}[leftmargin=0.55in]
    \item Une partie contextuelle services/produits achetés,acquittés ou non, acomptes règles, montant, date, heures etc
    \item Une partie prestataire avec ses information et son logo (cf.\ref{ss:sdu})
    \item Une partie client avec ses informations (cf.\ref{ss:cpc})
    \end{enumerate}
    \subsubsection{Saisir informations facture}
    Le système doit permettre de saisir les données  de la facture à partir de
    boîtes de dialogue
    \subsubsection{Charger informations facture}
    Le système doit permettre d'entrer les informations liées au prestataire et au client à partir
    de profil utilisateur et de profil client
    \subsubsection{Enregistrement dans la base de données}
    Lors de la création d'une facture le système permet la création une entrée dans la base de données
    \subsection{Lire facture}
    \subsubsection{Recherche d'une facture}
    Le système doit permettre d'interroger la base de données pour retrouver
    un devis à partir de ses informations (date, nom client ...)
    \subsection{Mettre à jour facture} \label{ss:majf}
    Le système doit permettre la mise a jour du devis.
    \subsection{Exporter facture}
    Le système doit permettre de sortir la facture sous forme de \emph{PDF}
    \subsection{Appliquer style à facture}
    Le système permet d'appliquer différents styles à la facture
    \subsection{Gérer les acomptes}
    Le système doit permettre de modifier la facture (voir section~\ref{ss:majf}) et soustraire les montants acquittés.
    
    
\section{Logiciel}
    \subsection{Manuel}
    \subsubsection{Installation}
    Le système dispose d'un manuel d'installation 
    \subsubsection{Langue}
    Le système doit permettre de modifier la langue par défaut du logicielle
    \subsubsection{Utilisation}
    Le système dispose d'un manuel d'utilisation
    \subsection{Programmation}
    Le système est codé en python associé a une base de donnée (SQLite)
    \subsubsection{Bibliothèques utilisées}
    Bibliothèque IHM GTK
    \subsection{Accès}
    Le système aura accès au profil utilisateur ainsi qu'à une base de données répertoriant les clients et les factures
    
\end{document}
